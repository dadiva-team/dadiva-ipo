%
% Capítulo 1
%
\chapter{Introduction} \label{cap:intro}
Blood donation services play a vital role in the healthcare systems of nations worldwide, serving as a cornerstone of public health initiatives. In Portugal, the establishment of the Blood National Institute (Instituto Nacional do Sangue) in 1958 marked the inception of formal coordination of transfusion medicine. This institution, evolving over more than five decades, culminated in the establishment of the Portuguese Blood and Transplantation Institute (Instituto Português do Sangue e da Transplantação, IPST) in 2012 \cite{IPST_Historia}.

Throughout this historical trajectory, blood donation services have undergone substantial organizational reforms aimed at ensuring the safety of both donors and recipients. However, the donor screening process has seen limited evolution despite these systemic changes.

The "Council Recommendation of 29 June 1998 on the suitability of blood and plasma donors and the screening of donated blood in the European Community" \cite{eu-29-June-1998} underscores the importance of gathering information from potential donors through written questionnaires. Although the specifics of these questionnaires may vary among Member States, their primary objective remains consistent: to identify common risk behaviors and diseases.

According to the 2022 Transfusion Activity and the Portuguese Hemovigilance System Report \cite{IPST:2023:Report}, Portugal recorded 306,796 blood donations from 203,287 donors, with 373,209 donor registrations during the same period. Notably, the main reason for the temporary suspension of blood donations is low hemoglobin levels, followed by recent travel to high-risk regions and engagement in behaviors associated with increased health risks.

Institutions like the Portuguese Oncology Institute (Instituto Português de Oncologia, IPO) in Lisbon, which contributed 1.88\% of total blood donations in 2022, still rely on traditional, paper-based questionnaires for donor screening. However, this manual process, coupled with the need for cross-referencing against guidelines provided by IPST, is susceptible to inefficiencies and errors. Such inefficiencies may contribute to reduced donor adherence and suboptimal health outcomes.

In partnership with Lisbon's IPO this project endeavors to address these challenges by developing a digital platform. The platform aims to provide donors with a comprehensive digital questionnaire encompassing both standard and relevant sub-questions pertinent to the screening process. For healthcare professionals, the platform will offer streamlined access to donor responses alongside information regarding potential health risks. Additionally, administrators will have tools to manage user accounts, questionnaire structures, and information regarding drug/disease interactions with blood donation.

By reducing the need for additional questions during screening consultations, this platform seeks to enhance donor participation. This is particularly crucial given the observed decline in donor numbers and donations from 2013 to 2022, amounting to a decrease of over 30,000 donors and 50,000 donations. Through these efforts, we aim to foster greater engagement with blood donation initiatives, thus contributing to the broader health and well-being of our community.

The main challenge with this project is regulatory compliance, particularly given our team's limited expertise in this domain and, to confront this challenge, our development strategy prioritizes the creation of adaptable functionalities designed to meet a broad range of regulatory requirements. Additionally, maintaining close collaboration with Lisbon's IPO will afford us invaluable guidance, ensuring our platform aligns with established frameworks and standards. By taking these proactive measures, we aim to navigate regulatory complexities effectively and develop a robust, compliant solution that can be tailored to the needs of blood donation services.

\section{IPO Collaboration}
Our partnership with the Portuguese Oncology Institute (IPO) has been characterized by ongoing collaboration and close communication, with monthly discussions to ensure the success of the project. Every feature of the digital platform has been thoroughly discussed with IPO, allowing us to align the development process with their practical needs and regulatory requirements. These regular consultations have provided invaluable guidance, ensuring that the platform meets the operational realities of blood donation services while adhering to the highest healthcare standards. By integrating IPO’s expertise into each stage of development, we’ve created a solution that not only digitizes the donor screening process but also enhances efficiency, accuracy, and donor engagement.


\section{Report Organization}

This report is structured into 7 chapters. In chapter \ref{cap:problem_description} we describe the problem and proposed solution. In chapter \ref{cap:architecture} we go over the architecture of the proposed solution and the technologies used. In chapter \ref{cap:data_model} we decompose the complete Entity-Relationship diagram into smaller groups and elaborate on what each entity represents and the logic behind their relationships.In chapter \ref{cap:frontend_implementation} and \ref{cap:backend_implementation} we elaborate the implementation details for the frontend and backend respectively. Finally in chapter \ref{cap:future} we reflect on what was achieved with this project, the encountered challenges, lessons learned and future development.
%
% Secção 1.1
%
%\section{Nome da secção deste capítulo} \label{sec11}
%
%Texto da secção. Na figura~\ref{fig:logotipo} mostra-se o logótipo do ISEL. Em \cite{wiki:bigdata2019} encontra várias referências para o assunto. O artigo \cite{6547630} é o mais popular conforme indicação do IEEE. Logo a seguir aparece \cite{6824752}. A identificação das referências deve ser melhorada.
%
%% Colocar uma figura
%\begin{figure}[h]
%\begin{center}
%\resizebox{100mm}{!}{\includegraphics{./figures/logoISEL.png}}
%\end{center}
%\caption{Legenda da figura com o logótipo do ISEL.}\label{fig:logotipo}
%\end{figure}
%
%Continuação do texto depois do parágrafo que refere a figura.
%
%
%%
%% Secção 1.2
%%
%\section{A segunda secção deste capítulo} \label{sec12}
%Na segunda secção deste capítulo, vamos abordar o enquadramento,
%o contexto e as funcionalidades.
%
%%
%% Secção 1.2.1
%%
%\subsection{A primeira sub-secção desta secção} \label{sec121}
%As sub-secções são úteis para mostrar determinados conteúdos de forma
%organizada. Contudo, o seu uso excessivo também não contribui para a facilidade
%de leitura do documento.
%
%%
%% Secção 1.2.2
%%
%\subsection{A segunda sub-secção desta secção} \label{sec122}
%Esta é a segunda sub-secção desta secção, a qual termina aqui.
%
%
%%
%% Secção 1.3
%%
%\section{Organização do documento} \label{sec13}
%O restante relatório encontra-se organizado da seguinte forma.