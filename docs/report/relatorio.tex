% Classe do documento e parâmetros gerais.
\documentclass[a4paper,openright,twoside,11pt]{report}

% Packages utilizadas e respetivos parâmetros.
\usepackage[utf8]{inputenc}

\usepackage{lipsum} % gerador de texto
\usepackage{graphicx}
\usepackage{url}
\usepackage[Algoritmo]{algorithm}
\usepackage{algorithmicx}
\usepackage{algpseudocode}
\usepackage{float}
\renewcommand{\algorithmicrequire}{\textbf{Dados: }}
\renewcommand{\algorithmicensure}{\textbf{Resultado: }}

% Definições das dimensões das páginas
\setlength{\textheight}{24.00cm}
\setlength{\textwidth}{15.50cm}
\setlength{\topmargin}{0.35cm}
\setlength{\headheight}{0cm}
\setlength{\headsep}{0cm}
\setlength{\oddsidemargin}{0.25cm}
\setlength{\evensidemargin}{0.25cm}

%\renewcommand{\baselinestretch}{1}

% Página inicial (capa)
\title{
   \vspace{-50mm}
   \begin{minipage}[l]{\textwidth}
      \hspace{-20mm}\resizebox{75mm}{!}{\includegraphics{./figures/logoISEL.png}}\\
   \end{minipage}\\[10mm]
   \textbf{\Huge DADIVA IPO}\\
   \textbf{D}igital \textbf{A}id and \textbf{D}onor \textbf{I}nformation \textbf{V}erification \textbf{A}pplication for \textbf{IPO}\\[5mm]
}

% Nome dos autores (um por linha)
\author{
\begin{tabular}{cr}
             & Francisco Medeiros  \\
             & Luís Macário \\
             & Ricardo Pinto \\[50mm]
\end{tabular}}

\date{
\begin{tabular}{ll}
  {Orientadores:} & Filipe Freitas, ISEL \\
                  & João Pereira, COFIDIS\\
\end{tabular}\\[10mm]
% Deixar o indicador respetivo em função da versão do relatório.
Relatório do projeto realizado no âmbito de Projecto e Seminário\\
Licenciatura em Engenharia Informática e de Computadores\\[20mm]
*Junho* de 2024}


\begin{document}
\pagenumbering{roman}
\thispagestyle{empty}
\maketitle

\baselineskip 18pt % line spacing: 12pt for single, 18pt for 1 1/2, and 24pt for double spacing

\newpage
\thispagestyle{empty}
% Fim da contracapa

% Página com identificação completa (número e nome) e assinaturas do(s) estudante(s) e do(s) orientador(es)
\cleardoublepage
\setcounter{page}{1}
\begin{center}
\textsc{\LARGE Instituto Superior de Engenharia de Lisboa}\\[20mm]

\textbf{\Huge DADIVA IPO}\\
\textbf{D}igital \textbf{A}id and \textbf{D}onor \textbf{I}nformation \textbf{V}erification \textbf{A}pplication for \textbf{IPO}\\[15mm]

\begin{tabular}{rl}
  46331  & Francisco Rodrigues Medeiros\\[10mm]
           & \rule{75mm}{0.5pt}\\[5mm]
  47671  & Luís Miguel Teixeira Macário\\[10mm]
           & \rule{75mm}{0.5pt}\\
  47673  & Ricardo Parreira Pinto\\[10mm]
           & \rule{75mm}{0.5pt}\\
\end{tabular}\\[10mm]

\begin{tabular}{rl}
  Orientadores: & Filipe Freitas, ISEL\\[10mm]
                & \rule{75mm}{0.5pt}\\[5mm]
                & Joao Pereira, COFIDIS\\[10mm]
                & \rule{75mm}{0.5pt}\\
\end{tabular}\\[10mm]

Relatório do projeto realizado no âmbito de Projecto e Seminário\\
Licenciatura em Engenharia Informática e de Computadores\\[20mm]
*Junho* de 2024\\
\end{center}

% Página de resumo em Português
\cleardoublepage
\chapter*{Resumo}
Texto do resumo.
Breve descrição do projeto, dos resultados importantes e das conclusões: o objetivo é dar ao leitor uma visão global do projeto (não deve exceder uma página).

% Página de resumo em Inglês
\cleardoublepage
\chapter*{Abstract}
The Instituto Português de Oncologia (IPO) in Lisbon currently employs a manual system for managing blood donor information. This involves donors completing a pre-donation form on paper, followed by a medical interview where a doctor assesses eligibility based on the form and additional verbal questions. This manual process of handling and verifying medical and medication details is prone to errors and inefficiency.

The proposed project aims to digitalize the blood donation process at IPO. This includes creating a digital version of the pre-donation form and developing a system to manage and cross-reference medication and pathology data. The digital system will allow for easy updating, customization, and retrieval of information. By automating the form and data handling, the project seeks to reduce errors associated with manual data management and decrease the overall time required for the donation process, thereby streamlining both donor signup and triage procedures.

%{\bf Palavras-chave:} lista de palavras-chave separadas por ;.

%% Página de agradecimentos
%\cleardoublepage
%\chapter*{Agradecimentos}
%Texto dos agradecimentos. É opcional.\\

% Geração do índice de conteúdos
\cleardoublepage
\tableofcontents \cleardoublepage

% Geração do índice de figuras e de tabelas
%\listoffigures \cleardoublepage
%\listoftables \cleardoublepage

% Reiniciar a numeração de páginas
\setcounter{page}{1}
\pagenumbering{arabic}

% Capitulo 1
%
% Capítulo 1
%
\chapter{Introduction} \label{cap:intro}
Blood donation services play a vital role in the healthcare systems of nations worldwide, serving as a cornerstone of public health initiatives. In Portugal, the establishment of the Blood National Institute (Instituto Nacional do Sangue) in 1958 marked the inception of formal coordination of transfusion medicine. This institution, evolving over more than five decades, culminated in the establishment of the Portuguese Blood and Transplantation Institute (Instituto Português do Sangue e da Transplantação, IPST) in 2012 \cite{IPST_Historia}.

Throughout this historical trajectory, blood donation services have undergone substantial organizational reforms aimed at ensuring the safety of both donors and recipients. However, the donor screening process has seen limited evolution despite these systemic changes.

The "Council Recommendation of 29 June 1998 on the suitability of blood and plasma donors and the screening of donated blood in the European Community" \cite{eu-29-June-1998} underscores the importance of gathering information from potential donors through written questionnaires. Although the specifics of these questionnaires may vary among Member States, their primary objective remains consistent: to identify common risk behaviors and diseases.

According to the 2022 Transfusion Activity and the Portuguese Hemovigilance System Report \cite{IPST:2023:Report}, Portugal recorded 306,796 blood donations from 203,287 donors, with 373,209 donor registrations during the same period. Notably, the main reason for the temporary suspension of blood donations is low hemoglobin levels, followed by recent travel to high-risk regions and engagement in behaviors associated with increased health risks.

Institutions like the Portuguese Oncology Institute (Instituto Português de Oncologia, IPO) in Lisbon, which contributed 1.88\% of total blood donations in 2022, still rely on traditional, paper-based questionnaires for donor screening. However, this manual process, coupled with the need for cross-referencing against guidelines provided by IPST, is susceptible to inefficiencies and errors. Such inefficiencies may contribute to reduced donor adherence and suboptimal health outcomes.

In partnership with Lisbon's IPO this project endeavors to address these challenges by developing a digital platform. The platform aims to provide donors with a comprehensive digital questionnaire encompassing both standard and relevant sub-questions pertinent to the screening process. For healthcare professionals, the platform will offer streamlined access to donor responses alongside information regarding potential health risks. Additionally, administrators will have tools to manage user accounts, questionnaire structures, and information regarding drug/disease interactions with blood donation.

By reducing the need for additional questions during screening consultations, this platform seeks to enhance donor participation. This is particularly crucial given the observed decline in donor numbers and donations from 2013 to 2022, amounting to a decrease of over 30,000 donors and 50,000 donations. Through these efforts, we aim to foster greater engagement with blood donation initiatives, thus contributing to the broader health and well-being of our community.

The main challenge with this project is regulatory compliance, particularly given our team's limited expertise in this domain and, to confront this challenge, our development strategy prioritizes the creation of adaptable functionalities designed to meet a broad range of regulatory requirements. Additionally, maintaining close collaboration with Lisbon's IPO will afford us invaluable guidance, ensuring our platform aligns with established frameworks and standards. By taking these proactive measures, we aim to navigate regulatory complexities effectively and develop a robust, compliant solution that can be tailored to the needs of blood donation services.

\section{IPO Collaboration}
Our partnership with the Portuguese Oncology Institute (IPO) has been characterized by ongoing collaboration and close communication, with monthly discussions to ensure the success of the project. Every feature of the digital platform has been thoroughly discussed with IPO, allowing us to align the development process with their practical needs and regulatory requirements. These regular consultations have provided invaluable guidance, ensuring that the platform meets the operational realities of blood donation services while adhering to the highest healthcare standards. By integrating IPO’s expertise into each stage of development, we’ve created a solution that not only digitizes the donor screening process but also enhances efficiency, accuracy, and donor engagement.


\section{Report Organization}

This report is structured into 7 chapters. In chapter \ref{cap:problem_description} we describe the problem and proposed solution. In chapter \ref{cap:architecture} we go over the architecture of the proposed solution and the technologies used. In chapter \ref{cap:data_model} we decompose the complete Entity-Relationship diagram into smaller groups and elaborate on what each entity represents and the logic behind their relationships.In chapter \ref{cap:frontend_implementation} and \ref{cap:backend_implementation} we elaborate the implementation details for the frontend and backend respectively. Finally in chapter \ref{cap:future} we reflect on what was achieved with this project, the encountered challenges, lessons learned and future development.
%
% Secção 1.1
%
%\section{Nome da secção deste capítulo} \label{sec11}
%
%Texto da secção. Na figura~\ref{fig:logotipo} mostra-se o logótipo do ISEL. Em \cite{wiki:bigdata2019} encontra várias referências para o assunto. O artigo \cite{6547630} é o mais popular conforme indicação do IEEE. Logo a seguir aparece \cite{6824752}. A identificação das referências deve ser melhorada.
%
%% Colocar uma figura
%\begin{figure}[h]
%\begin{center}
%\resizebox{100mm}{!}{\includegraphics{./figures/logoISEL.png}}
%\end{center}
%\caption{Legenda da figura com o logótipo do ISEL.}\label{fig:logotipo}
%\end{figure}
%
%Continuação do texto depois do parágrafo que refere a figura.
%
%
%%
%% Secção 1.2
%%
%\section{A segunda secção deste capítulo} \label{sec12}
%Na segunda secção deste capítulo, vamos abordar o enquadramento,
%o contexto e as funcionalidades.
%
%%
%% Secção 1.2.1
%%
%\subsection{A primeira sub-secção desta secção} \label{sec121}
%As sub-secções são úteis para mostrar determinados conteúdos de forma
%organizada. Contudo, o seu uso excessivo também não contribui para a facilidade
%de leitura do documento.
%
%%
%% Secção 1.2.2
%%
%\subsection{A segunda sub-secção desta secção} \label{sec122}
%Esta é a segunda sub-secção desta secção, a qual termina aqui.
%
%
%%
%% Secção 1.3
%%
%\section{Organização do documento} \label{sec13}
%O restante relatório encontra-se organizado da seguinte forma.

% Capitulo 2
%
% Capítulo 2
%
\chapter{Problem Description} \label{cap:problem_description}

Current blood donation workflow faces a set of challenges like screening time for more complex cases, since higher complexity cases may require cross-checking information about drug and pathology interaction, a process that, beyond being time-consuming,  may lead to imprecisions with severe consequences. Currently upon form changes the previously printed forms are disregarded, this process can be expedited by supporting a digital form that can be easily updated, helping IPO reduce its paper consumption.

These challenges can be met by employing a dynamic form, in digital format, that shows relevant follow up questions according to the potential donor's answers, thus collecting relevant information, that would otherwise need to be obtained during the medical screening.
This solution raises a set of questions such as:
\begin{itemize}
	\item What data structure is appropriate to describe the form's structure and flow/logic - 
	the questions order, possible answer values, what answers trigger or suppress follow-up questions;
	\item How will the form's rule be enforced in a way that doesn't force code implementation changes upon form structure changes - the frontend should be able to show and compute various forms and its unfeasible to change the frontend implementation upon every form structure change.
\end{itemize}

Upon form submission, the information supplied by the potential donor, or automatically obtained, can be automatically cross-checked against IPST guidelines for drug and pathology interaction with blood donation.
This solution raises a set of questions such as:
\begin{itemize}
	\item How are the potential donor's drug and disease information validated - the number of available drugs and possible diseases might be to great for real time validation, when the user is inputting that information into the form;
	\item Are the IPST guidelines available in a machine readable format that make it feasible to be cross-checked against the form's answers - to our knowledge, the guidelines are available in pdf and printed format, sometimes drugs/pathologies are individually mentioned and sometimes grouped in a family (ie there's no mention of aspirin in the 2022 manual, being replaced by Non Steroidal Anti Inflammatory, the family of drugs this medication belongs to).
\end{itemize}

The digital form structure and flow, pathology/drug interaction, and terms of service information should be updatable in the back-office. 

This solution raises a set of questions such as:
\begin{itemize}
	\item How can the form structure and flow be visualized intuitively- the user changing the form shouldn’t need to know anything about its implementation but still be able to identify and change its structure and flow;
	\item How will the drug/pathology interaction be updated- will a user manually insert information in the platform or can this information be requested via a web-service. 
\end{itemize}

Beyond these specific challenges the platform will have to employ multiple types of users, each with a given set of accesses, there are multiple ways of implementing role-based access control, each with pros and cons.

\section{Proposed Solution}

In order to solve the challenges listed above, we have developed DADIVA IPO.

DADIVA IPO is a web platform that allows blood donation services to decrease the screening time of blood donation candidates via a digital, updateable and dynamic form as well as automatic interaction verification.

It is intended as an alternative to the current, and less versatile, paper form used by blood donation services in Portugal, such as Lisbon's IPO.

\subsection{Functional Requirements}
\begin{itemize}
	\item Donors should be able to quickly fill out a digital pre-donation form. The form should be adequate according to the current law, adaptable, and depend on the donor’s answers.
	
	\item Doctors should be able to find all relevant data on pathology and/or medication interactions with the donation in a digital format.
	
	\item Doctors and administrators should be able to access a back office used for customizing the pre-donation form and for updating the pathology and/or medication interaction information. The back office should also allow for user management.
	
	\item Google-like search and results by relevance - Search should be as simple as possible. There may be a need to increase the number of filters, but this complexity should be hidden. The results returned should be sorted based on relevance.
\end{itemize}

\subsection{Non-Functional Requirements}
\begin{itemize}
	\item Intuitive user experience through a simple and practical user interface.
	
	\item Responsive design that ensures a good user experience both on desktop and mobile.
	
	\item Complete and thorough documentation.
	
	\item Unit and integration testing with sufficient coverage to ensure confidence that the system is working without flaws.
	
	\item Good software engineering practices to ensure the fast development of the system.
\end{itemize}

\subsection{Optional Features}
\begin{itemize}
	\item After filling out the pre-donation form, the system could automatically check if the donor had any vaccinations and/or prescriptions that could be medically relevant. It would require integration with the SNS, and/or Infarmed systems.
	\item The medical interview may be based on a pre-analysis, with the system having already identified possible risk vectors and logical incongruencies that better assist the doctor when deciding on accepting or refusing the donor.
	\item It is possible that the IPST has already implemented a digital system to maintain pathology and medication interaction information. If so, it would be possible to integrate this into our system, so that this information does not have to be manually updated.
	\item Users can authenticate using the Digital Mobile Key (CMD). It would require integration with the AMA (Administrative Modernization Agency) systems.
\end{itemize}

\pagebreak

\subsection{Use Cases}\label{sec:use_cases}
With the requirements listed above, we have identified the use cases that the platform
shall support. A use case is a written description of how users will perform tasks on
a system. It outlines, from the user's perspective, the behavior of the system as it
responds to a request. This approach attempts to predict the users of the platform, their allowed actions and objectives, and how the platform should respond to each action.
The use cases are divided into three categories, each representing one type of user.
The Donor use case is presented in Figure ~\ref{fig:donor_use_case}. The donor user can request the current form and can submit their form responses.

\begin{figure}[H]
	\begin{center}
		\resizebox{100mm}{!}{\includegraphics[trim={0 540 70 0},clip]{./figures/useCases.pdf}}
	\end{center}
	\caption{Donor use case.}\label{fig:donor_use_case}
\end{figure}

After a donor submits their form responses a doctor user will be able to access their answers by searching by the user's unique id as presented in Figure~\ref{fig:doctor_use_case}.

\begin{figure}[H]
	\begin{center}
		\resizebox{100mm}{!}{\includegraphics[trim={0 330 70 220},clip]{./figures/useCases.pdf}}
	\end{center}
	\caption{Doctor use case.}\label{fig:doctor_use_case}
\end{figure}

Furthermore the doctor user is able to search for pathology/medication interactions to resolve any inquiries that might appear during the screening.

Finally the administrator user can update the form structure and flow, update the interaction information and manage the platform's users.
The administrator use case is presented in Figure ~\ref{fig:administrator_use_case}.

\begin{figure}[H]
	\begin{center}
		\resizebox{100mm}{!}{\includegraphics[trim={0 0 0 450},clip]{./figures/useCases.pdf}}
	\end{center}
	\caption{Administrator use case.}\label{fig:administrator_use_case}
\end{figure}

% Capitulo 3
%
% Capítulo 4
%
\chapter{Architecture} \label{cap:architecture}

This chapter provides an overview of the system’s components and their interactions.
It outlines the capabilities of the project and presents the architecture, entities, and
implementation blueprint that have been designed and developed.

\section{Overview}

Figure ~\ref{fig:architecture} presents a diagram illustrating the main components of the system and their interactions. The system consists of a backend application (server-side) and a frontend application (client-side).
The backend comprises a collection of services, responsible for data manipulation and storage database.
The frontend is composed of a web application, that facilitates user interaction. Communication between the frontend and the backend is achieved through a REST API, utilizing the HTTP protocol [24].

\begin{figure}[H]
	\begin{center}
		\resizebox{150mm}{!}{\includegraphics{./figures/Architecture.png}}
	\end{center}
	\caption{Application Architecture.}\label{fig:architecture}
\end{figure}

\section{Form Data Model}

The first approach to solve de dynamic form challenge was to use a data structure formed by main questions and sub-questions, example presented in Figure ~\ref{fig:old_form}, where a main question can only be answered with boolean values, and one of those values triggers the display of a sub-question which has a certain type of response, such as boolean, dropdown for known multiple answers, and text to accept user text input.


\begin{figure}[hbt!]
	\begin{center}
		\resizebox{150mm}{!}{\includegraphics{./figures/oldForm.pdf}}
	\end{center}
	\caption{First Form Data Structure.}\label{fig:old_form}
\end{figure}

This approach has some drawbacks, such as the fact that it disables the possibility of supressing further questions, hence not adhering to the principle of creating a generic and adaptable solution, and mixes questions and rules in the sub-question.

Upon further discussion we settled on using a more complex data structure , exemplified in Figure ~\ref{fig:new_form}, composed by a list of questions and a list of rules.

Each question has an id, the text that composes it, the type of response (boolean, text and dropdown) and can have options that lists all the possibles values for a multiple(dropdown) response.

Each rule has conditions, which can be "any","all" or "not", so that, when any, all or none of the conditions are met an event is triggered, which can be to show or hide a question, the question targeted by the event is identified by the id, supplied via the params field.


\begin{figure}[htbp]
	\begin{center}
		{\includegraphics[width=\textwidth,height=\textheight,keepaspectratio]{./figures/newForm.pdf}}
	\end{center}
	\caption{Final Form Data Structure.}\label{fig:new_form}
\end{figure}
\FloatBarrier


\section{Frontend Application}

The frontend application is composed of a web application, which is responsible for the interaction between the users and the backend. This application provides a simple and intuitive interface for the user to interact with the system, allowing donor users to fill out the current form, doctor users to search for pathology and medication interaction with blood donation and request form answers of a given user and administrator users to customize the current form, update the pathology and medication interaction information and manage the user.
This application is divided into multiple pages and components, and has a service
layer that is responsible for communicating with the backend application through the
REST API.

\section{Services}

The backend application is composed of a set of services, responsible for data manipulation and storage. Each service is associated with a given domain and is independent from the remaining services, allowing for ease of future updates.
The services communicate with the database, in which the various data models are divided into specific indexes, such as:

\begin{itemize}
	\item /form: stores all the forms ;
	\item /submissions: stores all the user form responses ;
	\item /users: stores all the users.
\end{itemize}

The system is composed of the following services:

\begin{itemize}
	\item form: responsible for form management, such as, creation, requests, submission, editing and deletion;
	\item search: responsible for medication and pathology interaction information management;
	\item users: responsible for user management, such as, registration, login, deletion and role management.
\end{itemize}

\subsection{Form Services}

The form service is responsible for managing the form resources.
Figure ~\ref{fig:form_services} is a diagram that shows the architecture of the form services.

\begin{figure}[htbp]
	\begin{center}
		\resizebox{150mm}{!}{\includegraphics{./figures/formServices.pdf}}
	\end{center}
	\caption{Final Form Data Structure.}\label{fig:form_services}
\end{figure}


% Capitulo 4
\include{capexemplos}

% Capitulo 5
\include{captestes}

% Referências
\bibliographystyle{unsrt}
\bibliography{referencias}
\addcontentsline{toc}{chapter}{Refer\^{e}ncias}

% Apêndices (opcional)
\appendix
\include{apendiceex}

\end{document} 