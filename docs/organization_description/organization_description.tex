% Classe do documento e parâmetros gerais.
\documentclass[a4paper,openright,twoside,11pt]{article}

% Packages utilizadas e respetivos parâmetros.
\usepackage[utf8]{inputenc}

\usepackage{lipsum} % gerador de texto
\usepackage{graphicx}
\usepackage{url}
\usepackage[Algoritmo]{algorithm}
\usepackage{algorithmicx}
\usepackage{algpseudocode}
\usepackage[verbose]{placeins}
\renewcommand{\algorithmicrequire}{\textbf{Dados: }}
\renewcommand{\algorithmicensure}{\textbf{Resultado: }}
\usepackage{listings}
\usepackage{xcolor}
\usepackage{hyperref}

% Definições das dimensões das páginas
\setlength{\textheight}{24.00cm}
\setlength{\textwidth}{15.50cm}
\setlength{\topmargin}{0.35cm}
\setlength{\headheight}{0cm}
\setlength{\headsep}{0cm}
\setlength{\oddsidemargin}{0.25cm}
\setlength{\evensidemargin}{0.25cm}

% Página inicial (capa)
\title{
	\vspace{-50mm}
	\begin{minipage}[l]{\textwidth}
		\hspace{-20mm}\resizebox{75mm}{!}{\includegraphics{../report/figures/logoISEL.png}}\\
	\end{minipage}\\[10mm]
	\textbf{\Huge DADIVA IPO}\\
	\textbf{D}igital \textbf{A}id and \textbf{D}onor \textbf{I}nformation \textbf{V}erification \textbf{A}pplication for \textbf{IPO}\\[5mm]
}

% Nome dos autores (um por linha)
\author{
	\begin{tabular}{cr}
		& Francisco Medeiros \\
		& Luís Macário \\
		& Ricardo Pinto \\[50mm]
\end{tabular}}

\date{
	\begin{tabular}{ll}
		{Orientadores:} & Filipe Freitas, ISEL \\
		& João Pereira, COFIDIS\\
	\end{tabular}\\[10mm]
	% Deixar o indicador respetivo em função da versão do relatório.
	Descrição da organização do projeto realizado no âmbito de Projecto e Seminário\\
	Licenciatura em Engenharia Informática e de Computadores\\[20mm]
	*Junho* de 2024}
	
	
\begin{document}
\maketitle
\newpage

\section {\LARGE Project Repository}

DADIVA IPO platform's implementation and documentation are available through a GitHub repository available in \url{https://github.com/dadiva-team/dadiva-ipo}.

\section {\LARGE Usage}
In it's current stage to use the platform a user will first need to clone the GitHub repository.

Install PostgreSQL and have it running, available in \url{https://www.postgresql.org/download/}.

Install .NET, guides for multiple operating systems can be found in \url{https://learn.microsoft.com/en-us/dotnet/core/install/}.

Install Node.js, an installation guide can be found in \url{https://nodejs.org/en/learn/getting-started/how-to-install-nodejs}.

After installing all the needed software.

Run the "npm install" command in the \path{dadiva-ipo\code\DadivaWeb} directory followed by the "npm start", to run the frontend application.

Run the "dotnet ef migrations add InitialCreate" followed by "dotnet run --launch-profile https" command in the \path{dadiva-ipo\code\DadivaAPI\DadivaAPI} directory.

The frontend application will be accessible in \url{http://localhost:8000/}

\section {\LARGE Resources}
\begin{itemize}
	\item Project Proposal
	\item Project Presentation
	\item Project Report
	\item Poster
	\item GitHub Repository
\end{itemize}

\end{document}
