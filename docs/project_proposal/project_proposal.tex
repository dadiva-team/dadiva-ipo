\documentclass[a4paper,11pt]{article}
\usepackage{biblatex}
\usepackage{graphicx}

\addbibresource{references.bib}
% pdflatex

% redefines the page margins
\setlength{\textheight}{24.00cm}
\setlength{\textwidth}{15.50cm}
\setlength{\topmargin}{0.35cm}
\setlength{\headheight}{0cm}
\setlength{\headsep}{0cm}
\setlength{\oddsidemargin}{0.25cm}

\title{%
\includegraphics{logoISEL}
D\'{A}DIVA IPO\\\large Digital Aid and Donor Information Verification Application for IPO
}

\author{
\begin{tabular}{c}
	Francisco Medeiros, No 46631, e-mail: a46631@alunos.isel.pt, tel.: 913192828\\
	Luís Macário, No.47671, e-mail: a47671@alunos.isel.pt, tel.: 933694131\\
	Ricardo Pinto, No.47673, e-mail: a47673@alunos.isel.pt, tel.: 927111044\\
\end{tabular}}

\date{
\begin{tabular}{ll}
  {Supervisors:} & Filipe Freitas, e-mail: filipe.freitas@isel.pt \\
                 & João Pereira, e-mail: joaomiguel.pereira@cofidis.pt, Cofidis\\
\end{tabular}\\
\vspace{5mm}
\today}

\begin{document}

\maketitle

\tableofcontents

\section{Introduction}
Today, the professionals at the Instituto Português de Oncologia (IPO) in Lisbon, within the blood donor service, store and cross-check donor information in physical formats.

Currently, a donor arrives at IPO and is asked to fill and sign a paper pre-donation form. After filling out the form, the donor is then taken to a medical exam, where the doctor will, using the form, ask some questions and do some exams to determine the donor's eligibility. If the donor has a pathology and/or is currently taking medication, this could be a cause for non-eligibility. Currently, the doctor has to check these interactions in a paper format.

This project aims to digitalize the pre-donation form and the medication and/or pathology interaction information. These should be easily updated and customizable, and the pre-donation form should be adaptable to the donor’s answers.

The objective would be to significantly decrease the possibilities of mistakes that storing/consulting information in a paper format brings and decrease the time it takes to donate blood by fast-tracking the signup and triage processes.

\section{System Requirements}

\subsection{Functional Requirements}
\begin{itemize}
	\item Donors should be able to quickly fill out a digital pre-donation form. The form should be adequate according to the current law, adaptable, and depend on the donor’s answers.
	
	\item Doctors should be able to find all relevant data on pathology and/or medication interaction with the donation in a digital format.
	
	\item Doctors and administrators should be able to access a back office used for customizing the pre-donation form and for updating the pathology and/or medication interaction information. The back office should also allow for user management.
	
	\item Google-like search and results by relevance - Search should be as simple as possible. There may be a need to increase the number of filters, but this complexity should be hidden. The results returned should be sorted based on relevance.
\end{itemize}

\subsection{Non-Functional Requirements}
\begin{itemize}
	\item Intuitive user experience through a simple and practical user interface.
	
	\item Responsive design that ensures a good user experience both on desktop and mobile.
	
	\item Complete and thorough documentation.
	
	\item Unit and integration testing with sufficient coverage to ensure confidence that the system is working without flaws.
	
	\item Good software engineering practices to ensure the fast development of the system.
\end{itemize}

\subsection{Optional Features}
\begin{itemize}
	\item After filling out the pre-donation form, the system could automatically check if the donor had any vaccinations and/or prescriptions that could be medically relevant. It would require integration with the IPO system, SNS, and/or Infarmed.
	\item It is possible that the IPST has already implemented a digital system to maintain pathology and medication interaction information. If so, it would be possible to integrate this into our system, so that this information does not have to be manually updated.
	\item Users can authenticate using the Digital Mobile Key (CMD). It would require integration with the AMA (Administrative Modernization Agency) systems.
\end{itemize}

\section{Technologies}
We plan to use ASP.NET \cite{aspnet} to build the Web API, following DDD (Domain Driven Design) and using the Minimal API \cite{minimalAPI} architecture. We will use Node.js \cite{nodejs} and React \cite{react} to deliver the front-end experience. We will use ElasticSearch \cite{elasticsearch} to store the donor’s data and the contextualized information. ElasticSearch will be used due to its searching capabilities, returning search results based on a relevance algorithm.

\section{Risks}
Some factors that might influence our speed on the project development are:
\begin{enumerate}
	\item C\# \cite{csharp} and Asp.NET are technologies that we’re not experienced in, resulting in some time being needed for adaptation.

	\item We are not experienced in building a UI/UX that has been deployed and used by end users. It could be a challenge to create an intuitive and engaging user interface that meets the expectations of our users.

	\item Difficulties defining the project scope, since the actual project depends on the IPO's needs.

	\item If we end up having to integrate into the existing software, that is something that we have not done, and it could present as a challenge.
\end{enumerate}

\begin{figure}
\centering
\includegraphics[width=\textwidth,height=\textheight,keepaspectratio]{gantt.png}
\caption{Provisory Gantt Chart}
\end{figure}

\printbibliography[heading=bibintoc]

\end{document}